\documentclass{article}
\usepackage[a4paper, margin=1in]{geometry}
\usepackage{graphicx}
\usepackage{titling}
\usepackage{hyperref}
\usepackage{listings}
\usepackage{xcolor}
\usepackage{dirtree}

\definecolor{codegreen}{rgb}{0,0.6,0}
\definecolor{codegray}{rgb}{0.5,0.5,0.5}
\definecolor{codepurple}{rgb}{0.58,0,0.82}
\definecolor{backcolour}{rgb}{0.95,0.95,0.92}

\lstdefinestyle{mystyle}{
    backgroundcolor=\color{backcolour},   
    commentstyle=\color{codegreen},
    keywordstyle=\color{magenta},
    numberstyle=\tiny\color{codegray},
    stringstyle=\color{codepurple},
    basicstyle=\ttfamily\footnotesize,
    breakatwhitespace=false,         
    breaklines=true,                 
    captionpos=b,                    
    keepspaces=true,                 
    numbers=left,                    
    numbersep=5pt,                  
    showspaces=false,                
    showstringspaces=false,
    showtabs=false,                  
    tabsize=2
}

\lstset{style=mystyle}

\title{E-Voting System Project Report}
\author{Gemini}
\date{\today}

\begin{document}

\maketitle

\begin{abstract}
This report provides a detailed overview of the E-Voting System, a PHP-based web application. It covers the project structure, core functionalities, database schema, and key code implementations. The system is designed to allow users to register, vote for political parties, and for administrators to manage the election process.
\end{abstract}

\section{Introduction}
The E-Voting System is a web-based application built with procedural PHP and a MySQL database. It provides a platform for digital voting, aiming to create a secure and straightforward process for voters and administrators. The system features role-based access control for voters, candidates, and admins, ensuring a separation of duties.

\section{Project Structure}
The project is organized into several directories and PHP files. The following is a tree-view of the project structure:

\dirtree{
.1 .
.2 admin_candidates.php.
.2 admin_dashboard.php.
.2 admin_districts.php.
.2 admin_parties.php.
.2 campaigns.php.
.2 cand_dashboard.php.
.2 create_candidate.php.
.2 edit_profile.php.
.2 index.php.
.2 logout.php.
.2 national_results.php.
.2 register.php.
.2 results.php.
.2 run_migration.php.
.2 submit_vote.php.
.2 vote.php.
.2 assets.
.3 css.
.4 style.css.
.3 images.
.2 includes.
.3 auth_session.php.
.3 db.php.
.2 uploads.
}

\section{Core Components}
This section describes the key files and their roles in the application.

\subsection{Database Connection (\texttt{includes/db.php})}
This file is responsible for establishing a connection to the MySQL database named \texttt{evoting_system}. It creates a \texttt{\$conn} object that is used globally for all database queries.

\subsection{User Authentication (\texttt{index.php}, \texttt{register.php}, \texttt{includes/auth_session.php})}
User registration is handled by \texttt{register.php}, which securely hashes user passwords using \texttt{password_hash}. The \texttt{index.php} file serves as the login page, verifying credentials with \texttt{password_verify} and establishing a user session. Role-based redirection is performed upon successful login. The \texttt{auth_session.php} script is included in protected pages to ensure only authenticated users can access them.

\subsection{Voting Process (\texttt{vote.php}, \texttt{submit_vote.php})}
The \texttt{vote.php} page is the main interface for voters. It checks if the user has already voted. If not, it displays the political parties available in the user's electoral district. The actual voting is handled by \texttt{submit_vote.php}, which uses a database transaction to ensure atomicity. It records the vote and updates the user's status to prevent double voting.

\subsection{Admin Dashboard (\texttt{admin_dashboard.php} and related files)}
The admin dashboard provides administrators with tools to manage the election. This includes managing political parties (\texttt{admin_parties.php}), districts (\texttt{admin_districts.php}), and candidates (\texttt{admin_candidates.php}).

\section{Database Schema}
The database schema is inferred from the PHP code. The main tables are:

\begin{itemize}
    \item \texttt{users}: Stores user information, including credentials, role, and voting status.
    \item \texttt{political_parties}: Contains details about the political parties.
    \item \texttt{electoral_districts}: Stores the different electoral districts.
    \item \texttt{candidates}: Holds information about candidates.
    \item \texttt{party_votes}: Aggregates the vote counts for each party in each district.
\end{itemize}

\subsection{Inferred `party_votes` Table Structure}
Based on \texttt{submit_vote.php}, the \texttt{party_votes} table is likely structured as follows:
\begin{lstlisting}[language=SQL]
CREATE TABLE party_votes (
    district_id INT NOT NULL,
    party_id INT NOT NULL,
    vote_count INT DEFAULT 0,
    PRIMARY KEY (district_id, party_id)
);
\end{lstlisting}

\section{Key Code Snippets}

\subsection{Database Connection}
The following code from \texttt{includes/db.php} establishes the database connection.
\begin{lstlisting}[language=PHP]
<?php
$servername = "localhost";
$username = "root";
$password = "";
$dbname = "evoting_system";

// Create connection
$conn = new mysqli($servername, $username, $password, $dbname);

// Check connection
if ($conn->connect_error) {
    die("Connection failed: " . $conn->connect_error);
}
?>
\end{lstlisting}

\subsection{Vote Submission Logic}
The core logic for submitting a vote from \texttt{submit_vote.php} uses a transaction to ensure data integrity.
\begin{lstlisting}[language=PHP]
// Start a transaction
$conn->begin_transaction();

try {
    // Increment the vote count for the party in the district
    $stmt1 = $conn->prepare(
        "INSERT INTO party_votes (district_id, party_id, vote_count)
         VALUES (?, ?, 1)
         ON DUPLICATE KEY UPDATE vote_count = vote_count + 1"
    );
    $stmt1->bind_param("ii", $district_id, $party_id);
    $stmt1->execute();

    // Mark the user as having voted
    $stmt2 = $conn->prepare(
        "UPDATE users SET has_voted = 1 WHERE id = ?"
    );
    $stmt2->bind_param("i", $user_id);
    $stmt2->execute();

    // If both queries succeed, commit the transaction
    $conn->commit();

    echo "Vote submitted successfully!";

} catch (mysqli_sql_exception $exception) {
    // If any query fails, roll back the transaction
    $conn->rollback();
    
    echo "Error submitting vote: " . $exception->getMessage();
}
\end{lstlisting}


\end{document}
